% Team Note of National Potato.

\documentclass[landscape, 8pt, a4paper, oneside, twocolumn]{extarticle}

\usepackage[compact]{titlesec}
\titlespacing*{\section}
{0pt}{0px plus 1px minus 0px}{-2px plus 0px minus 0px}
\titlespacing*{\subsection}
{0pt}{0px plus 1px minus 0px}{0px plus 3px minus 3px}

\setlength{\columnseprule}{0.4pt}
\pagenumbering{arabic}

\usepackage{kotex}

\usepackage[left=0.8cm, right=0.8cm, top=2cm, bottom=0.3cm, a4paper]{geometry}
\usepackage{amsmath}
\usepackage{ulem}
\usepackage{amssymb}
\usepackage{minted}
\usepackage{color, hyperref}
\usepackage{indentfirst}
\usepackage{enumitem}

\usepackage{fancyhdr}
\usepackage{lastpage}
\pagestyle{fancy}
\lhead{KNU - National Potato.}
\rhead{Page \thepage  \ of \pageref{LastPage} }
\fancyfoot{}

\headsep 0.2cm

\setminted{breaklines=true, tabsize=2, breaksymbolleft=}
\usemintedstyle{perldoc}

\setlength\partopsep{-\topsep -\parskip}

\title{Team Note of National Potato}
\author{lim551(unkoob)}
\date{Compiled on \today}

\newcommand{\revised}{Should be \textcolor{red}{\textbf{revised}}.}
\newcommand{\tested}{Should be \textcolor{red}{\textbf{tested}}.}
\newcommand{\added}{Should be \textcolor{red}{\textbf{added}}.}
\newcommand{\WIP}{\textcolor{red}{\textbf{WIP}}}
\begin{document}

{
	\Large

	\maketitle

\tableofcontents
}
\thispagestyle{fancy}
\pagebreak

\textcolor{red}{\textbf{ALL BELOW HERE ARE USELESS IF YOU READ THE STATEMENT WRONG}}

\section{Basic}
\subsection {Preset}
\begin{minted}{cpp}
#include <bits/stdc++.h>
#include <ext/pb_ds/assoc_container.hpp>
#include <ext/pb_ds/tree_policy.hpp>
using namespace std;
using namespace __gnu_pbds;
template<class key, class value, class cmp = less<key>>
using treemap = tree<key, value, less<int>, rb_tree_tag, tree_order_statistics_node_update>; // key, val, comp, implements, node raw
template<class key, class cmp = less<key>>
using treeset = tree<key, null_type, cmp, rb_tree_tag, tree_order_statistics_node_update>;

#ifdef LOCAL_BOOKNU
#define debug(...) cerr << "[" << #__VA_ARGS__ << "]:", debug_out(__VA_ARGS__)
#else
#define debug(...) 42
#endif

// ........................macro.......................... //
#define FOR(i, f, n) for(int (i) = (f); (i) < (int)(n); ++(i))
#define RFOR(i, f, n) for(int (i) = (f); (i) >= (int)(n); --(i))
#define pb push_back
#define emb emplace_back
#define fi first
#define se second
#define ENDL '\n'
#define sz(A) (int)(A).size()
#define ALL(A) A.begin(), A.end()
#define UNIQUE(c) (c).resize(unique(ALL(c)) - (c).begin())
#define next next9876
#define prev prev1234
typedef pair<int, int> ii;
typedef pair<int, ii> iii;
typedef vector<int> vi;
typedef vector<vi> vvi;
typedef vector<ii> vii;
typedef vector<vii> vvii;
typedef long long i64;
typedef unsigned long long ui64;
inline int getidx(const vi& ar, int x) { return lower_bound(ALL(ar), x) - ar.begin(); }
inline i64 GCD(i64 a, i64 b) { i64 n; if(a < b) swap(a, b); while(b != 0) { n = a % b; a = b; b = n; } return a; }
inline i64 LCM(i64 a, i64 b) { if(a == 0 || b == 0) return GCD(a, b); return a / GCD(a, b) * b; }
inline i64 CEIL(i64 n, i64 d) { return n / d + (i64)(n % d != 0); }
inline i64 ROUND(i64 n, i64 d) { return n / d + (i64)((n % d) * 2 >= d); }
const i64 MOD = 1e9+7;
inline i64 POW(i64 a, i64 n) {
	assert(0 <= n);
	i64 ret;
	for(ret = 1; n; a = a*a%MOD, n /= 2) { if(n%2) ret = ret*a%MOD; }
	return ret;
}
template <class T> ostream& operator<<(ostream& os, vector<T> v) {
	os << "[";
	int cnt = 0;
	for(auto vv : v) { os << vv; if(++cnt < v.size()) os << ","; }
	return os << "]";
}
template <class T> ostream& operator<<(ostream& os, set<T> v) {
	os << "[";
	int cnt = 0;
	for(auto vv : v) { os << vv; if(++cnt < v.size()) os << ","; }
	return os << "]";
}
template <class L, class R> ostream& operator<<(ostream& os, pair<L, R> p) { return os << "(" << p.fi << "," << p.se << ")"; }
void debug_out() { cerr << endl; }
template <typename Head, typename... Tail> void debug_out(Head H, Tail... T) { cerr << " " << H, debug_out(T...); }
// ....................................................... //


void input() {
	
}

int solve() {
	
	return 0;
}

// ................. main .................. //
void execute() {
	input(), solve();
}

int main(void) {
#ifdef LOCAL_BOOKNU
	freopen("__IO/input.txt", "r", stdin);
	// freopen("__IO/out.txt", "w", stdout);
#endif
	cin.tie(0), ios_base::sync_with_stdio(false);
	execute();
	return 0;
}
// ......................................... //
\end{minted}
\subsection {Credits}
\begin{itemize}[noitemsep,nolistsep]
	\item cki86201, zigui, PavelKunyavskiy
	\item \url{https://gist.github.com/msg555/4963794}
	\item \url{https://github.com/niklasb/contest-algos/blob/master/convex_hull/dynamic.cpp}
	\item \url{https://github.com/jaehyunp/stanfordacm}
	\item \url{https://github.com/tzupengwang/PECaveros/blob/master/codebook/graph/BorrowedGeneralWeightedMatching.cpp}
	\item \url{https://github.com/tzupengwang/PECaveros/blob/master/codebook/math/DiscreteKthsqrt.cpp}
	\item \url{http://www-math.mit.edu/~etingof/groups.pdf}
\end{itemize}
\end{document}