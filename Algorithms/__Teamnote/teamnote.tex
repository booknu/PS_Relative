% Team Note of National Potato.

\documentclass[landscape, 8pt, a4paper, oneside, twocolumn]{extarticle}

\usepackage[compact]{titlesec}
\titlespacing*{\section}
{0pt}{0px plus 1px minus 0px}{-2px plus 0px minus 0px}
\titlespacing*{\subsection}
{0pt}{0px plus 1px minus 0px}{0px plus 3px minus 3px}

\setlength{\columnseprule}{0.4pt}
\pagenumbering{arabic}

\usepackage{kotex}

\usepackage[left=0.8cm, right=0.8cm, top=2cm, bottom=0.3cm, a4paper]{geometry}
\usepackage{amsmath}
\usepackage{ulem}
\usepackage{amssymb}
\usepackage{minted}
\usepackage{color, hyperref}
\usepackage{indentfirst}
\usepackage{enumitem}

\usepackage{fancyhdr}
\usepackage{lastpage}
\pagestyle{fancy}
\lhead{KNU - National Potato.}
\rhead{Page \thepage  \ of \pageref{LastPage} }
\fancyfoot{}

\headsep 0.2cm

\setminted{breaklines=true, tabsize=2, breaksymbolleft=}
\usemintedstyle{perldoc}

\setlength\partopsep{-\topsep -\parskip}

\title{Team Note of National Potato}
\author{lim551(unkoob), qse1346(vidm), ventulus95}
\date{Compiled on \today}

\newcommand{\revised}{Should be \textcolor{red}{\textbf{revised}}.}
\newcommand{\tested}{Should be \textcolor{red}{\textbf{tested}}.}
\newcommand{\added}{Should be \textcolor{red}{\textbf{added}}.}
\newcommand{\WIP}{\textcolor{red}{\textbf{WIP}}}
\begin{document}

{
	\Large

	\maketitle

\tableofcontents
}
\thispagestyle{fancy}
\pagebreak

\textcolor{red}{\textbf{ALL BELOW HERE ARE USELESS IF YOU READ THE STATEMENT WRONG}}

\section{Network Flow}
\subsection {Dinic's Algorithm}
\begin{minted}{cpp}
/*
	1. build level graph with bfs ( O(E) )
	2. flow blocking flow in level graph ( O(VE) )
	In each step, level grows at least by 1, and eventually grows upto O(V)
	So total time complexity is O(V^2E)
*/
struct dinic { // O(V^2 E)
	struct edg { int v, c, r; };
	int n;
	vi dis, itr;
	vector<vector<edg>> g;
	dinic(int n) : n(n), g(n, vector<edg>()), dis(n), itr(n) {  }
	void addedge(int u, int v, int c) {
		g[u].pb({ v, c, sz(g[v]) });
		g[v].pb({ u, 0, sz(g[u])-1 });
	}
	bool bfs(int s, int t) { // build level graph
		dis.assign(n, 0), itr.assign(n, 0);
		queue<int> q;
		q.push(s);
		dis[s] = 1;
		while(q.size()) {
			int u = q.front(); q.pop();
			for(auto& [v, c, r] : g[u]) {
				if(c > 0 && !dis[v]) {
					dis[v] = dis[u] + 1;
					q.push(v);
				}
			}
		}
		return dis[t] > 0;
	}
	int dfs(int u, int t, int f) { // get blocking flow
		if(u == t) return f;
		for( ; itr[u] < g[u].size(); itr[u]++) {
			auto& [v, c, r] = g[u][itr[u]];
			if(c > 0 && dis[v] == dis[u] + 1) {
				int w = dfs(v, t, min(f, c));
				if(w) {
					g[u][itr[u]].c -= w;
					g[v][r].c += w;
					return w;
				}
			}
		}
		return 0;
	}
	i64 nflow(int s, int t) { // network flow
		i64 ret = 0;
		while(bfs(s, t)) {
			int r;
			while(r = dfs(s, t, 2e9)) ret += r, debug("-----");
		}
		return ret;
	}
};
\end{minted}
\subsection {Hofcroft-Karp Bipartite Matching}
\begin{minted}{cpp}
/*
    - alternating path: path consists of (x, y, x, y, x) for X = { x | x is matched edge}, Y = { y | y is not mathed edge } 
    - augmenting path: path consists of (y, x, y, x, y)
    - if augmenting path exists, we can match one more edge with fliping matched state (x, y, x, y, x)

    For maximaum matching A, B
    1. lv[0] = { v | v in A and v is not matched }
    2. starting from lv[0] vertices, get alternating path with bfs
    3. starting from lv[0] vertices, get augmenting path with dfs

    * min cover: selecting minimum vertices to cover all edges
    * max independent set: selecting maximum vertices not connected with edge
    * V - min cover = max independent set
*/
struct hofcroft {
	int n, m;
	vi dis, l, r, vis, chk;
	vvi g;
	hofcroft(int n, int m) : n(n), m(m), g(n, vi()) {  }
	void addedge(int u, int v) { g[u].pb(v); }
	bool bfs() { // build alternating path starts from lv[0] nodes
		queue<int> q;
		bool ok = 0;
		dis.assign(n, 0);
		FOR(u, 0, n) {
			if(l[u] == -1 && !dis[u]) {
				q.push(u);
				dis[u] = 1;
			}
		}
		while(q.size()) {
			int u = q.front(); q.pop();
			for(int v : g[u]) {
				if(r[v] == -1) ok = 1; // v is not matched
				else if(!dis[r[v]]) { // if v is matched, u>v>r[v] can be path
					dis[r[v]] = dis[u] + 1;
					q.push(r[v]);
				}
			}
		}
		return ok;
	}
	bool dfs(int u) { // find augmenting path and flip it!
		if(vis[u]) return 0; // augmenting path start/end with non-matched vertices
		vis[u] = 1;
		for(int v : g[u]) {
			if(r[v] == -1 || (dis[r[v]] == dis[u] + 1 && dfs(r[v]))) {
				l[u] = v; r[v] = u;
				return 1;
			}
		}
		return 0;
	}
	int match() { // bipartite match
		l.assign(n, -1);
		r.assign(m, -1);
		int ret = 0;
		while(bfs()) {
			vis.assign(n, 0);
			FOR(u, 0, n) if(l[u] == -1 && dfs(u)) ++ret;
		}
		return ret;
	}
	void rdfs(int u) { // dfs matched
		if(chk[u]) return;
		chk[u] = 1;
		for(int v : g[u]) {
			chk[v + n] = 1;
			rdfs(r[v]);
		}
	}
	vi getcover() { // get min cover vertices
		match();
		chk.assign(n+m, 0);
		FOR(u, 0, n) if(l[u] == -1) rdfs(u);
		vi ret;
		FOR(u, 0, n) if(!chk[u]) ret.pb(u);
		FOR(u, n, n+m) if(chk[u]) ret.pb(u);
		return ret;
	}
};
\end{minted}
\subsection {Minimum Cost Maximum Flow}
\begin{minted}{cpp}
const int WINF = 0x3fffffff, FINF = 0x3fffffff; // weight/flow inf
struct mcmf {
	struct edg { int v, c, r, w; };
	int n;
	vi dis, par, peg;
	vector<bool> inq;
	vector<vector<edg> > g;
	mcmf(int n) : n(n), g(n, vector<edg>()), par(n), peg(n) {  }
	void addedge(int u, int v, int c, int w) {
		g[u].pb({ v, c, sz(g[v]), w });
		g[v].pb({ u, 0, sz(g[u])-1, -w });
	}
	bool spfa(int s, int t) {
		dis.assign(n, WINF);
		inq.assign(n, 0);
		queue<int> q;
		dis[s] = 0;
		inq[s] = 1;
		q.push(s);
		bool ok = 0;
		while(q.size()) {
			int u = q.front(); q.pop();
			if(u == t) ok = 1;
			inq[u] = 0;
			FOR(eidx, 0, g[u].size()) {
				auto [v, c, r, w] = g[u][eidx];
				if(c > 0 && dis[v] > dis[u] + w) {
					dis[v] = dis[u] + w;
					par[v] = u;
					peg[v] = eidx;
					if(!inq[v]) {
						inq[v] = 1;
						q.push(v);
					}
				}
			}
		}
		return ok;
	}
	ii flow(int s, int t) { // return (max_flow, min_cost)
		int cost = 0, flow = 0;
		while(spfa(s, t)) {
			int cur = FINF;
			for(int u = t; u != s; u = par[u]) cur = min(cur, g[par[u]][peg[u]].c);
			for(int u = t; u != s; u = par[u]) {
				int r = g[par[u]][peg[u]].r;
				g[par[u]][peg[u]].c -= cur;
				g[u][r].c += cur;
			}
			flow += cur;
			cost += dis[t] * cur;
		}
		return { flow, cost };
	}
};
\end{minted}
\subsection {Ford-Fulkerson Algorithm}
\begin{minted}{cpp}
// Caution: All vertices' idx > 0 (par[S] = 0)
const int MAXND = 500, S = 1, T = 2, INF = 0x3fffffff;
struct nflow {
	struct edge {
		edge* rev; 
		int v, c, f; // need initialized
		edge(int v, int c) : v(v), c(c), f(0) { }
		int res() { return c-f; }
		int flow(int x) { f += x, rev->f -= x; }
	};
	vector<edge*> g[MAXND];
	int par[MAXND];
	edge* pedg[MAXND];
	nflow() {
		FOR(i, 0, MAXN) g[i] = vector<edge*>();
	}
	void addedge(int u, int v, int c) { 
		edge *uv = new edge(v, c), *vu = new edge(u, 0);
		uv->rev = vu, vu->rev = uv;
		g[u].pb(uv), g[v].pb(vu);
	}
	i64 maxflow() {
		i64 ret = 0;
		while(true) {
			memset(par, 0, sizeof(par));
			queue<int> q;
			q.push(S);
			par[S] = S;
			while(q.size()) {
				int u = q.front(); q.pop();
				for(auto e : g[u]) {
					if(e->res() && !par[e->v]) {
						q.push(e->v);
						par[e->v] = u;
						pedg[e->v] = e;
						if(e->v == T) break;
					}
				}
				if(par[T]) break;
			}
			if(!par[T]) break;
			int flow = INF;
			for(int u = T; u != S; u = par[u]) flow = min(flow, pedg[u]->res());
			for(int u = T; u != S; u = par[u]) pedg[u]->flow(flow);
			ret += flow;
		}
		return ret;
	}
};
\end{minted}
\subsection {Naive Bipartite Match}
\begin{minted}{cpp}
const int MAXN = 5e2+10;
int vis[MAXN], ato[MAXN], bto[MAXN];
bool dfs(int u) {
	if(vis[u]) return 0;
	vis[u] = 1;
	for(int v : g[u]) {
		if(bto[v] == -1 || dfs(bto[v])) {
			ato[u] = v;
			bto[v] = u;
			return 1;
		}
	}
	return 0;
}
int bimatch() {
	memset(ato, -1, sizeof(ato)), memset(bto, -1, sizeof(bto));
	int ret = 0;
	FOR(u, 0, n) {
		memset(vis, 0, sizeof(vis));
		if(dfs(u)) ++ret;
	}
	return ret;
}
\end{minted}
\section{Graph}
\subsection {2-SAT + SCC}
\begin{minted}{cpp}
/*
	2-SAT: (A || B) && (C || D) && (E || F) ...
	1. X || Y = !X -> Y, !Y -> X  (Proposition)
		False: T -> F, True: Others
	2. !X, X in same SCC: no solution
	3. For every SCC, each node in same SCC must have same flag (if both T, F exists in same SCC, T->F exists)
	4. Assign False to (Don't have in edge & Unassigned node) and erase node
		- sort nodes topologically, iterate nodes with assigning False to var if var is unassigned
			- !X node: X = True, X node: X = False
*/
struct sat2 {
	struct tarjan {
		int n, ncnt, scnt;
		vi scc, dis;
		vvi g;
		stack<int> sta;
		tarjan(int n) : n(n), g(n, vi()) { } // n: number of variables (NOT NODES!)
		void addedge(int u, int v) { g[u].pb(v); } // directed graph
		int f(int u) {
			int ret = dis[u] = ncnt++;
			sta.push(u);
			for(int v : g[u]) {
				if(dis[v] == -1) ret = min(ret, f(v));
				else if(scc[v] == -1) ret = min(ret, dis[v]);
			}
			if(ret == dis[u]) {
				while(1) {
					int t = sta.top(); sta.pop();
					scc[t] = scnt;
					if(t == u) break;
				}
				++scnt;
			}
			return ret;
		}
		vi& get_scc() {
			ncnt = scnt = 0;
			scc = dis = vi(n, -1);
			sta = stack<int>();
			FOR(i, 0, n) if(dis[i] == -1) f(i);
			dis.clear();
			return scc;
		}
	};
	int n;
	vi res;
	tarjan tj;
	sat2(int n) : n(n), tj(2*n) { }
	int nd(int u, int neg) { return u + neg*n; } // var u's node
	int neg(int u) { return (u+n)%(2*n); } // ~u
	void addedge(int u, int nu, int v, int nv) { // add (X || Y) clauses
		u = nd(u, nu), v = nd(v, nv);
		tj.addedge(neg(u), v);
		tj.addedge(neg(v), u); 
	} 
	vi& solve() { // return solved vars, if no solution return vi()
		vi& scc = tj.get_scc();
		FOR(u, 0, n) if(scc[u] == scc[u+n]) return res;
		res.assign(n, -1);
		vi ord(2*n);
		FOR(i, 0, 2*n) ord[i] = i;
		sort(ALL(ord), [&](int u, int v) { return scc[u] > scc[v]; });
		FOR(i, 0, 2*n) {
			int u = ord[i];
			if(res[u%n] == -1) res[u%n] = !(u<n);
		}
		return res;
	}
};
\end{minted}
\subsection {Cut-Vertex}
\begin{minted}{cpp}
// Find Bridge
const int MAXN = 3e5, INF = 0x7fffffff;
int ncnt, vid[MAXN], par[MAXN];
vii g[MAXN]; // (v, bridge flag)
int dfs(int u) {
	int ret = vid[u] = ncnt++;
	for(auto& e : g[u]) {
		int v = e.se, c = INF;
		if(par[u] == v) continue; // Tree edge
		if(vid[v] == -1) { // Tree Edge
			par[v] = u;
			c = min(c, dfs(v));
		} else c = min({ c, vid[v], vid[u] }); // Forward/Backward Edge
		if(c > vid[u]) e.fi = 1; // Bridge
		ret = min(ret, c);
	}
	return ret;
}
\end{minted}
\section{String}
\subsection {KMP Algorithm}
\begin{minted}{cpp}
string s, t;
vi getpi(const string& str) {
	int n = str.size(), len = 0;
	vi pi(n, 0);
	FOR(i, 1, n) {
		while(len && str[len] != str[i]) len = pi[len-1];
		if(str[len] == str[i]) pi[i] = ++len;
	}
	return pi;
}

vi kmp() {
	int n = s.size(), m = t.size(), len = 0;
	vi ret, pi = getpi(t);
	FOR(i, 0, n) {
		while(len && s[i] != t[len]) len = pi[len-1];
		if(s[i] == t[len] && ++len == m) ret.pb(i-len+1), len = pi[len-1];
	}
	return ret;
}
\end{minted}
\subsection {Rabin-Karp Algorithm}
\begin{minted}{cpp}
// f(p) = s[0] + s[1] * p + s[2] * p^2 + ... + s[n-1] * p^{n-1)
// h[i+1] = p * (h[i] - s[i] * s^(m-1)) + s[i+m]
// --> sub first character from hash > hash degree up > add last character to hash
const i64 MUL = 232153, MD = 1012924417; // be careful for MD not MOD
void mod(i64& x) { x %= MD; if(x < 0) x += MD; }
vi rabin(string& s, string& t) { // return start indexs
	vi ret;
	i64 ht = 0, hs = 0, mul = 1;
	RFOR(i, sz(t)-1, 0) { // get t's hash
		mod(ht += mul * t[i] % MD);
		mod(mul = mul * MUL);
	}
	mul = 1;
	RFOR(i, sz(t)-1, 0) { // get s's hash for first sz(t) string
		mod(hs += mul * s[i] % MD);
		if(i != 0) mod(mul = mul * MUL); // mul must be p^{m-1)
	}
	if(hs == ht) ret.pb(0);
	FOR(i, sz(t), sz(s)) {
		mod(hs -= mul*s[i-sz(t)]%MD);
		mod(hs *= MUL);
		mod(hs += s[i]);
		if(hs == ht) ret.pb(i-sz(t)+1);
	}
	return ret;
}
\end{minted}
\subsection {Trie(Array)}
\begin{minted}{cpp}
const int MAX_NODE = 1e6+10;
int cld[MAX_NODE][30];
i64 cnt[MAX_NODE];
int ncnt = 1;
void push(const string& x) {
	int u = 0;
	FOR(i, 0, x.size()) {
		if(!cld[u][x[i]-'a']) cld[u][x[i]-'a'] = ncnt++;
		u = cld[u][x[i]-'a'];
	}
	++cnt[u];
}
void calc_back(int u) {
	FOR(i, 0, 30) {
		if(cld[u][i]) {
			calc_back(cld[u][i]);
			cnt[u] += cnt[cld[u][i]];
		}
	}
	cnt[u] %= MOD;
}
\end{minted}
\subsection {Aho-Corasick Algorithm}
\begin{minted}{cpp}
const int MAXNODE = 1e5+10, MAXC = 26, INITCHAR = 'a';
struct ahocorasick {
	int ncnt, t[MAXNODE][MAXC], f[MAXNODE], chk[MAXNODE];
	ahocorasick() : ncnt(0) { memset(t, 0, sizeof(t)), memset(f, 0, sizeof(f)), memset(chk, 0, sizeof(chk)); }
	void insert(const string& s) {
		int u = 0;
		for(auto i : s) {
			i -= INITCHAR;
			if(!t[u][i]) t[u][i] = ++ncnt;
			u = t[u][i];
		}
		chk[u] = 1; // end of string
	}
	void precalc() {
		queue<int> q;
		FOR(i, 0, MAXC) if(t[0][i]) q.push(t[0][i]);
		// calculdate fail, chk with bfs
		while(q.size()) {
			int x = q.front(); q.pop();
			FOR(i, 0, MAXC) {
				if(t[x][i]) {
					int u = x, p = f[u];
					while(p && !t[p][i]) p = f[p]; // find fail link
					u = t[u][i], p = t[p][i]; // goto original target node
					f[u] = p;
					if(chk[p]) chk[u] = 1;
					q.push(u);
				}
			}
		}
	}
	bool query(const string& s) {
		int u = 0;
		for(auto i : s) {
			i -= INITCHAR;
			while(u && !t[u][i]) u = f[u];
			if(chk[u = t[u][i]]) return true;
		}
		return false;
	}
};
\end{minted}
\subsection {Suffix Array + LCP}
\begin{minted}{cpp}
/*
	sa[i] = ordered suffix (suffix's start position)
	ord[i] = [i:]'s index in sa  (ord[sa[i]] = i)
	lcp[i] = longest common prefix length of two suffix [i-1:], [i:]
	
	LCP's Lemma
	1.	Two adjacent in SA suffixes' LCP is always bigger than which of non-adjacents
	2.	lcp(sa[i-1], sa[i]) = h, h >= 1 then
		lcp(sa[i-1]+1, sa[i]+1) = h-1
		
	So that lcp[sa[i]+1] >= h-1 because it is always bigger than lcp(sa[i-1]+1, sa[i]+1) by Lemma 1
	and by Lemma 2, lcp(sa[i-1]+1, sa[i]+1) = h-1
*/
struct sfxarray {
	int n;
	string& str;
	vi sa, lcp, ord;
	sfxarray(string& str) : str(str), n(str.size()) { }
	void getsa() {
		sa = ord = vi(n+1);
		FOR(i, 0, n) sa[i] = i, ord[i] = str[i]; ord[n] = 0;
		for(int t = 1; t <= n; t *= 2) {
			int sz = max(257, n+1);
			vi cnt, tmp;
			cnt = tmp = vi(sz, 0);
			FOR(i, 0, n) ++cnt[ord[min(n, i+t)]];
			FOR(i, 1, sz) cnt[i] += cnt[i-1];
			FOR(i, 0, n) tmp[--cnt[ord[min(n, i+t)]]] = i;
			cnt = vi(sz, 0);
			FOR(i, 0, n) ++cnt[ord[i]];
			FOR(i, 1, sz) cnt[i] += cnt[i-1];
			RFOR(i, n-1, 0) sa[--cnt[ord[tmp[i]]]] = tmp[i];
			tmp[sa[0]] = 1;
			FOR(i, 1, n) {
				int u = sa[i-1], v = sa[i];
				tmp[v] = tmp[u] + (ord[u] < ord[v] || ord[u+t] < ord[v+t]);
			}
			ord = tmp;
			if(ord[sa[n-1]] == n) break;
		}
		FOR(i, 0, n) --ord[i];
	}
	void getlcp() {
		lcp = vi(n, 0);
		for(int i = 0, len = 0; i < n; ++i, len = max(0, len-1)) {
			if(ord[i]) {
				for(int j = sa[ord[i]-1]; str[i+len] == str[j+len]; ++len);
				lcp[ord[i]] = len;
			}
		}
	}
	tuple<vi, vi, vi> build() { getsa(), getlcp(); return { sa, lcp, ord }; }
};
\end{minted}
\subsection {Manacher's Algorithm}
\begin{minted}{cpp}
const int MAXN = 1000005;
int aux[2 * MAXN - 1];
void solve(int n, int *str, int *ret){
	// *ret : number of nonobvious palindromic character pair
	for(int i=0; i<n; i++){
		aux[2*i] = str[i];
		if(i != n-1) aux[2*i+1] = -1;
	}
	int p = 0, c = 0;
	for(int i=0; i<2*n-1; i++){
		int cur = 0;
		if(i <= p) cur = min(ret[2 * c - i], p - i);
		while(i - cur - 1 >= 0 && i + cur + 1 < 2*n-1 && aux[i-cur-1] == aux[i+cur+1]){
			cur++;
		}
		ret[i] = cur;
		if(i + ret[i] > p){
			p = i + ret[i];
			c = i;
		}
	}
}
\end{minted}
\section{Math}
\subsection {Fraction}
\begin{minted}{cpp}
// dependency: GCD(i64 a, i64 b)
struct frac {
	i64 a, b;
	frac(i64 _a=0, i64 _b=1) : a(_a), b(_b) { if(a == 0 && b == 0) b = 1; assert(b != 0); relax(); }
	// Essential: Basic Operations
	void relax() { i64 g = GCD(abs(a), abs(b)); a /= g, b /= g; }
	frac operator + (const frac &ot) const { return { a * ot.b + ot.a * b, b * ot.b }; }
	frac operator - (const frac &ot) const { return { a * ot.b - ot.a * b, b * ot.b }; }
	frac operator * (const frac &ot) const { return { a * ot.a, b * ot.b }; }
	frac operator / (const frac &ot) const { return { a * ot.b, b * ot.a }; }
	frac operator - () { return { -a, b }; }
	// Essential: Basic Comparation
	bool operator == (const frac& ot) const { return a * ot.b == ot.a * b; }
	bool operator < (const frac& ot) const { return a * ot.b < ot.a * b; }
	bool operator <= (const frac& ot) const { return a * ot.b <= ot.a * b; }
	bool operator > (const frac& ot) const { return ot <= *this; }
	bool operator >= (const frac& ot) const { return ot < *this; }
	// Optional: Advanced Operations
	const frac& operator += (const frac &ot) { return *this = *this + ot; }
	const frac& operator -= (const frac &ot) { return *this = *this - ot; }
	const frac& operator *= (const frac &ot) { return *this = *this * ot; }
	const frac& operator /= (const frac &ot) { return *this = *this / ot; }
};
// fraction IO
ostream& operator<< (ostream& os, const frac& frac_x) { return os << frac_x.a << "/" << frac_x.b; }
istream& operator>> (istream& os, frac& frac_x) {
	os >> frac_x.a >> frac_x.b;
	frac_x.relax();
	return os;
}
\end{minted}
\subsection {Matrix}
\begin{minted}{cpp}
// Do not use this class as const
typedef i64 ELEM;
const ELEM MOD = 1e9+7; // If don't use MOD, set as 0x7fffffffffffffff
struct mat {
	int n, m;
	vector<vector<ELEM> > ar;
	// ----- constructor, assignment  ----- //
	mat(int n, int m, ELEM x = 0) : n(n), m(m), ar(n, vector<ELEM>(m, x)) {  }
	mat(int n = 0) : mat(n, n) {  }
	mat(const mat& o) { n = o.n, m = o.m, ar = o.ar; }
	mat(const vector<vector<ELEM>>& ar) : n(ar.size()), m(ar.size() ? ar[0].size() : 0), ar(ar) { }
	// ----- get field ----- //
	operator const vector<vector<ELEM> >& () const { return ar; }
	vector<ELEM>& operator[](int i) { return ar[i]; }
	const vector<ELEM>& operator[](int i) const { return ar[i]; }
	// ----- calculate ----- //
	mat pow(i64 x) const {
		assert(n == m && 0 <= x);
		mat a(*this), ret = eye(n);
		while(x) {
			if(x%2) ret = ret * a;
			a = a * a;
			x /= 2;
		}
		return ret;
	}
	mat operator * (const mat& o) const {
		assert(m == o.n);
		mat ret(n, o.m);
		FOR(i, 0, n) {
			FOR(j, 0, o.m) {
				FOR(k, 0, m) {
					ret[i][j] += ar[i][k] * o[k][j] % MOD;
					ret[i][j] %= MOD;
				}
			}
		}
		return ret;
	}
	mat operator + (const mat& o) const {
		assert(n == o.n && m == o.m);
		mat ret(n, m);
		FOR(i, 0, n) FOR(j, 0, n) ret[i][j] = (ar[i][j] + o[i][j]) % MOD;
		return ret;
	}
	mat operator - (const mat& o) const {
		assert(n == o.n && m == o.m);
		mat ret(n, m);
		FOR(i, 0, n) FOR(j, 0, n) ret[i][j] = (ar[i][j] - o[i][j]) % MOD;
		return ret;
	}
	mat operator * (const ELEM x) const {
		mat ret = ar;
		FOR(i, 0, n) FOR(j, 0, m) ret[i][j] = ret[i][j] * x % MOD;
		return ret;
	}
	mat operator / (const ELEM x) const {
		mat ret = ar;
		FOR(i, 0, n) FOR(j, 0, m) ret[i][j] = ret[i][j] / x % MOD;
		return ret;
	}
	const mat& operator - () {
		FOR(i, 0, n) FOR(j, 0, m) ar[i][j] = -ar[i][j];
		return *this;
	}
    // If use dp matrix as: state = state * dpmat
    // use rotated dpmat and horizontal state mat
	mat rotate() const {
		mat ret(m, n);
		FOR(i, 0, n) FOR(j, 0, m) ret[j][i] = ar[i][j];
		return ret;
	}
	static mat eye(const int size) {
		mat ret(size);
		FOR(i, 0, size) ret[i][i] = 1;
		return ret;
	}
	// return matrix dp
    // dp[i] = ar[0] * dp[i-n] + ar[1] * dp[i-n+1] + ... + ar[n-1] * dp[i-1]
	// data matrix br: br[0] = dp[i], br[1] = dp[i-1] ...
	// If DP equation contains constant, fix one element for constant
	static mat dpmat(const vector<ELEM>& ar) {
		int n = ar.size();
		mat ret(n, n);
		FOR(i, 0, n-1) ret[i][i+1] = 1; // transition prev dp values
		FOR(i, 0, n) ret[n-1][i] = ar[i]; // DP equation
		return ret;
	}
};
ostream& operator<<(ostream& os, const mat& v) { for(auto vv : (vector<vector<ELEM> >)v) os << vv << ENDL; return os; }
\end{minted}
\subsection {Miller-Rabin Test}
\begin{minted}{cpp}
namespace miller_rabin{ // O(logP)
	i64 mul(i64 x, i64 y, i64 mod){ return (__int128) x * y % mod; }
	i64 ipow(i64 x, i64 y, i64 p){
		i64 ret = 1, piv = x % p;
		while(y){
			if(y&1) ret = mul(ret, piv, p);
			piv = mul(piv, piv, p);
			y >>= 1;
		}
		return ret;
	}
	bool miller_rabin(i64 x, i64 a){
		if(x % a == 0) return 0;
		i64 d = x - 1;
		while(1){
			i64 tmp = ipow(a, d, x);
			if(d&1) return (tmp != 1 && tmp != x-1);
			else if(tmp == x-1) return 0;
			d >>= 1;
		}
	}
	bool isprime(i64 x){
		for(auto &i : {2, 3, 5, 7, 11, 13, 17, 19, 23, 29, 31, 37}){
			if(x == i) return 1;
			if(x > 40 && miller_rabin(x, i)) return 0;
		}
		if(x <= 40) return 0;
		return 1;
	}
}
\end{minted}
\subsection {Euler's Sieve}
\begin{minted}{cpp}
const int RANGE = 2e7;
int pn, spf[RANGE], pr[RANGE]; // spf[x] = min prime factor of x
void eulerSieve() {
	FOR(x, 2, RANGE) {
		if(!spf[x]) spf[x] = pr[pn++] = x; // if x is prime, spf[x] = x
		for(int j = 0; x*pr[j] < RANGE; ++j) {
			spf[x*pr[j]] = pr[j];
			if(x % pr[j] == 0) break;
		}
	}
}
\end{minted}
\subsection {Binomial Coefficient}
\begin{minted}{cpp}
// 1. c[n][r] = c[n-1][r-1] + c[n-1][r]
// 2. nCr = n! / ((n-r)! * r!)
const int RANGE = 1e6;
// precalc: n!, n!^(-1) --> O(NlogP)
void precalc_1() {
	i64 ftr[RANGE], iftr[RANGE];
	ftr[0] = iftr[0] = 1;
	FOR(i, 1, RANGE) {
		ft[i] = (ft[i-1] * (i64)i) % MOD;
		ift[i] = (ift[i-1] * POW((i64)i, MOD-2)) % MOD;
	}
}
// (n-1)!^(-1) =  n*n!^(-1) --> O(n+logP)
void precalc_2() {
	i64 ftr[RANGE], iftr[RANGE];
	ftr[0] = iftr[0] = 1;
	FOR(i, 1, RANGE) ftr[i] = ftr[i-1] * i;
	iftr[RANGE-1] = POW(ftr[RANGE-1], MOD-2);
	RFOR(i, RANGE-2, 0) iftr[i] = (i * iftr[i+1]) % MOD;
}
// inv(1) = 1, inv(1) = -floor(p/i) + inv(p%i) --> O(n)
void precalc_3() {
	i64 inv[RANGE+1], ftr[RANGE], iftr[RANGE];
	inv[1] = 1;
	FOR(i, 2, RANGE+1) {
		inv[i] = inv[MOD % i] - (MOD / i);
		if(inv[i] < 0) inv[i] += MOD;
		inv[i] %= MOD;
	}
}
\end{minted}
\subsection {Inclusion-Exclusion Principle}
\begin{minted}{cpp}
FOR(i, 1, (1 << n)) { // get Union(A, B, C, D ...)
    int bits = __builtin_popcount(i);
    if(bits % 2); // add to ans
    else; // sub to ans
}
\end{minted}
\subsection {Josephus Problem}
\begin{minted}{cpp}
/*	O(n)
	f(n, k) = last survived person for n-people, k-cycle
	
	< basic idea >
	except 1 element from f(n, k), then answer is f(n-1, k)
	but f(n-1, k) need to be repositioned to starting from kth's next person
	
	< 1-indexed >
	f(1, k) = 1
	f(n, k) = ((f(n-1, k) + k-1) % n) + 1
	< 0-indexed >
	f(1, k) = 0;
	f(n, k) = ((f(n-1, k) + k) % n)
*/
// O(KlogN) algorithm 
long long joseph (long long n,long long k) {
    if (n==1LL) return 0LL;
    if (k==1LL) return n-1LL;
    if (k>n) return (joseph(n-1LL,k)+k)%n;
    long long cnt=n/k;
    long long res=joseph(n-cnt,k);
    res-=n%k;
    if (res<0LL) res+=n;
    else res+=res/(k-1LL);
    return res;
}
\end{minted}
\section{Segment Tree}
\subsection {Fenwick Tree Tricks}
\begin{minted}{cpp}
struct fenwick {
	int n;
	vector<i64> t;
	void init(int _n) { n = _n, t = vector<i64>(n+1, 0); }
	void add(int u, i64 x) { for(++u; u < t.size(); u += (u&-u)) t[u] += x; }
	i64 sum(int u) {
        i64 ret = 0;
		for(++u; u > 0; u -= (u&-u)) ret += t[u];
		return ret;
	}
	i64 operator[](int u) {
		++u;
		int ret = t[u], p = u - (u&-u);
		--u;
		while(u != p) {
			ret -= t[u];
			u -= (u&-u);
		}
		return ret;
	}
	// Can use all elements are positive.
	// k >= 0
	// find x which ( sum[a[0]..a[x-1]] < k <= sum[a[0]..a[x]] )
	int lower(i64 k) {
		if(k < 0) return 0;
		int l = (1 << (8*sizeof(int) - __builtin_clz(n)) - 1);
		int u = 0;
		while(l > 0 && u <= n) {
			int tu = u + l;
			if(k > t[tu]) u = tu, k -= t[tu];
			do l >>= 1;
			while(l > 0 && u + l > n);
		}
		return u;
	}
	// find x which ( sum[a[0]..a[x-1]] <= k < sum[a[0]..a[x]] )
	int upper(i64 k) {
		if(k < 0) return 0;
		int l = (1 << (8*sizeof(int) - __builtin_clz(n)) - 1);
		int u = 0;
		while(l > 0 && u <= n) {
			int tu = u + l;
			if(k >= t[tu]) u = tu, k -= t[tu];
			do l >>= 1;
			while(l > 0 && u + l > n);
		}
		return u;
	}
};
\end{minted}
\subsection {Fenwick Tree 2D (Sparse)}
\begin{minted}{cpp}
struct segtree {
	vi ys[RANGE], t[RANGE];
	// Notify segtree update access on (x, y)
	void initpos(int x, int y) {
		for(++x; x < RANGE; x += (x&-x)) {
			ys[x].pb(y);
		}
	}
	// Execute after notifying (x, y)
	void init() {
		FOR(i, 0, RANGE) sort(ALL(ys[i])), UNIQUE(ys[i]), t[i].assign(ys[i].size()+1, 0);
	}
	// add (x, y) to c
	void add(int x, int y, int c) {
		for(++x; x < RANGE; x += (x&-x)) {
			for(int j = getidx(ys[x], y)+1; j < sz(t[x]); j += (j&-j)) {
				t[x][j] += c;
			}
		}
	}
	// partial sum of ([..x], [..y])
	int sum(int x, int y) {
		int ret = 0;
		for(++x; x > 0; x -= (x&-x)) {
			int j = getidx(ys[x], y);
			if(j == ys[x].size() || ys[x][j] > y) --j;
			for(++j; j > 0; j -= (j&-j)) {
				ret += t[x][j];
			}
		}
		return ret;
	}
} seg;

\end{minted}
\subsection {Fenwick Tree Range Update/Query}
\begin{minted}{cpp}
struct rfenwick { // using 2 basic fenwick tree
	fenwick ax, b;
	void init(int n) { ax.init(n), b.init(n); }
	void add(int u, i64 x) { b.add(u, x); }
	void add(int s, int e, i64 x) {
		ax.add(s, x);
		ax.add(e+1, -x);
		b.add(s, -x * (s-1));
		b.add(e+1, x * e);
	}
	i64 sum(int u) {
		return u * ax.sum(u) + b.sum(u);
	}
};
\end{minted}
\subsection {Segment Tree (Loop)}
\begin{minted}{cpp}
template<class T, class C>
struct segtree {
	int n;
	vector<T> t;
	void build(const vector<T>& ar) {
		n = ar.size();
		t.assign(n*2, 0);
		FOR(i, 0, n) t[n+i] = ar[i];
		RFOR(i, n-1, 1) t[i] = C()(t[i*2], t[i*2+1]);
	}
	void mod(int u, T k) {
		for(t[u += n] = k; u > 1; u /= 2) t[u/2] = C()(t[u], t[u^1]);
	}
	T query(int s, int e) {
		T ret = 0;
		for(s += n, e += n; s < e; s /= 2, e /= 2) {
			if(s & 1) ret = C()(ret, t[s++]);
			if(e & 1) ret = C()(ret, t[--e]);
		}
		return ret;
	}
};
\end{minted}
\subsection {Segment Tree 2D (Dense)}
\begin{minted}{cpp}
struct segtree {
	int n, m;
	vvi t;
	void init(const vvi& ar) {
		n = ar.size(), m = ar[0].size();
		t.assign(2*n, vi(2*m, 0));
		FOR(y, 0, n) { // push in leaf
			FOR(x, 0, m) {
				t[n+y][m+x] = ar[y][x];
			}
		}
		RFOR(y, 2*n-1, 1) { // construct
			RFOR(x, 2*m-1, 1) {
				if(y < n) t[y][x] = t[y*2][x] + t[y*2+1][x];
				if(x < m) t[y][x] = t[y][x*2] + t[y][x*2+1];
			}
		}
	}
	void modify(int y, int x, int c) {
		t[y + n][x + m] = c; // leaf update
		for(y += n; y > 0; y /= 2) {
			for(int x2 = x + m; x2 > 0; x2 /= 2) {
				if(y < n) t[y][x2] = t[y*2][x2] + t[y*2+1][x2];
				if(x2 < m) t[y][x2] = t[y][x2*2] + t[y][x2*2+1];
			}
		}
	}
	int query(int sy, int sx, int ey, int ex) {
		int ret = 0;
		for(sy += n, ey += n; sy < ey; sy /= 2, ey /= 2) {
			for(int x1 = sx + m, x2 = ex + m; x1 < x2; x1 /= 2, x2 /= 2) {
				if(sy&1) {
					if(x1&1) ret += t[sy][x1];
					if(x2&1) ret += t[sy][x2-1];
				}
				if(ey&1) {
					if(x1&1) ret += t[ey-1][x1];
					if(x2&1) ret += t[ey-1][x2-1];
				}
				if(x1&1) ++x1;
				if(x2&1) --x2;
			}
			if(sy&1) ++sy;
			if(ey&1) --ey;
		}
		return ret;
	}
};
\end{minted}
\subsection {Lazy Propagation}
\begin{minted}{cpp}
const int ST_MAX = 1<<21, lf = ST_MAX/2; 
struct segtree{
	i64 t[ST_MAX], d[ST_MAX];
	segtree(){ memset(t, 0, sizeof(t)), memset(d, 0, sizeof(d)); }
	void build(){ RFOR(i, lf-1, 1) t[i] = t[i*2]+ t[i*2+1]; } // !! BUILD !!
	void propagate(int u, int ns, int ne){
		if(!d[u]) return;
		if(u < lf){ // propagate to childs
			d[u*2] += d[u];
			d[u*2+1] += d[u];
		}
		t[u] += d[u] * (ne-ns); // update node
		d[u] = 0;
	}
	void add(int s, int e, int x){ add(s, e, x, 1, 0, lf); } // [s, e)
	void add(int s, int e, int x, int u, int ns, int ne){
		propagate(u, ns, ne);
		if(e <= ns || ne <= s) return;
		if(s <= ns && ne <= e){
			d[u] += x;
			propagate(u, ns, ne);
			return;
		}
		int mid = (ns+ne)/2;
		add(s, e, x, u*2, ns, mid), add(s, e, x, u*2+1, mid, ne);
		t[u] = t[u*2] + t[u*2+1];
	}
	i64 sum(int s, int e){ return sum(s, e, 1, 0, lf); } // [s, e)
	i64 sum(int s, int e, int u, int ns, int ne){
		propagate(u, ns, ne);
		if(e <= ns || ne <= s) return 0;
		if(s <= ns && ne <= e) return t[u];
		int mid = (ns+ne)/2;
		return sum(s, e, u*2, ns, mid) + sum(s, e, u*2+1, mid, ne);
	}
};
\end{minted}
\subsection {Persistent Segment Tree}
\begin{minted}{cpp}
struct node {
	int x;
	node *l, *r;
	node(int _x = 0, node* l = 0, node* r = 0) : x(_x), l(l), r(r) {  }
	node* addtree(int u, int c, int ns = 0, int ne = MAXN-1) {
		if(ns <= u && u <= ne) {
			if(ns == ne) return new node(x + c, 0, 0);
			int mid = (ns+ne)/2;
			return new node(x + c, l->addtree(u, c, ns, mid), r->addtree(u, c, mid+1, ne));
		}
		return this;
	}
	int query(int s, int e, int ns = 0, int ne = MAXN-1) {
		if(s <= ns && ne <= e) return x;
		if(ne < s || e < ns) return 0;
		int mid = (ns+ne)/2;
		return l->query(s, e, ns, mid) + r->query(s, e, mid+1, ne);
	}
} *root[MAXN+1];
\end{minted}
\subsection {Persistent Segment Tree (Array)}
\begin{minted}{cpp}
struct pst {
	i64 x[MAXN*LOGN];
	int l[MAXN*LOGN], r[MAXN*LOGN], tcnt;
	int base(int ns = 0, int ne = MAXN-1) { // make 0th tree
		int u = tcnt++;
		l[u] = u, r[u] = u;
	}
	int make(int idx, int c, int u, int ns = 0, int ne = MAXN-1) { // update from u-rooted
		if(idx < ns || ne < idx) return u;
		int v = tcnt++;
		if(ns == ne) x[v] = (x[u] + c) % MOD;
		else {
			int m = (ns+ne)/2;
			l[v] = make(idx, c, l[u], ns, m);
			r[v] = make(idx, c, r[u], m+1, ne);
			x[v] = (x[l[v]] + x[r[v]]) % MOD;
		}
		return v;
	}
	i64 query(int s, int e, int u, int ns = 0, int ne = MAXN-1) { // query from u-rooted
		if(s <= ns && ne <= e) return x[u];
		if(ne < s || e < ns) return 0;
		int m = (ns+ne)/2;
		return (query(s, e, l[u], ns, m) + query(s, e, r[u], m+1, ne)) % MOD; 
	}
};
\end{minted}
\subsection {HLD (Vertex)}
\begin{minted}{cpp}
/*
	HLD with costed vertex.

	usually (dfs_init, lca, decomposite, eidx, query) don't need to be changed.
	just modify (segtree, init_segs), and if segtree function changed modify (update, query_to)
*/
const int MAXN = 1e5+10, LOGN = 18, INF = 0x7fffffff;
struct hld_vtx {
	struct segtree {
		int n;
		vi t;
		void init(int _n) { n = _n; t.assign(2*n, INF); }
		void update(int u, int x) {
			for(t[u += n] = x; u > 1; u /= 2) t[u/2] = min(t[u], t[u^1]);
		}
		int query(int s, int e) {
			int ret = INF;
			for(s += n, e += n; s < e; s /= 2, e /= 2) {
				if(s&1) ret = min(ret, t[s++]);
				if(e&1) ret = min(ret, t[--e]);
			}
			return ret;
		}
	};	
	int n, rt;
	vi ssz, dep, hidx;
	vvi g, par, hvy;
	vector<segtree> segs;
	hld_vtx(vvi& g, int rt) : g(g), rt(rt), n(g.size()), ssz(n, 0), dep(n, 0), hidx(n, -1), par(n, vi(LOGN, 0)) { 
		par[rt][0] = rt;
		dfs_init(rt);
		decomposite(rt);
		init_segs();
	}
	void dfs_init(int u) { // initialize dfs info
		ssz[u] = 1;
		FOR(j, 1, LOGN) par[u][j] = par[par[u][j-1]][j-1];
		for(int v : g[u]) {
			if(par[u][0] == v) continue;
			par[v][0] = u;
			dep[v] = dep[u] + 1;
			dfs_init(v);
			ssz[u] += ssz[v];
		}
	}
	int lca(int u, int v) { // consider par[root] = root
		if(dep[u] < dep[v]) swap(u, v);
		int dif = dep[u] - dep[v];
		FOR(j, 0, LOGN) if(dif & (1<<j)) u = par[u][j];
		if(u != v) {
			RFOR(j, LOGN-1, 0) if(par[u][j] != par[v][j]) u = par[u][j], v = par[v][j];
			u = par[u][0];
		}
		return u;
	}
	void decomposite(int rt) { // decomposite tree
		queue<int> q;
		q.push(rt);
		while(q.size()) {
			int u = q.front(); q.pop();
			for(int v : g[u]) if(par[v][0] == u) q.push(v);
			int p = par[u][0];
			if(u != rt && ssz[u]*2 >= ssz[p]) { // extend h-path
				hidx[u] = hidx[p];
				hvy[hidx[u]].pb(u);
			} else { // create h-path
				hidx[u] = hvy.size();
				hvy.pb(vi());
				hvy[hidx[u]].pb(u);
			}
		}
	}
	void init_segs() { // initialize segtrees
		segs.assign(hvy.size(), segtree());
		FOR(i, 0, hvy.size()) segs[i].init(hvy[i].size()); // m nodes
	}
	int vidx(int u) { // get v's index in h-path
		return dep[u] - dep[hvy[hidx[u]][0]];
	}
	void update(int u, int x) { // update v's cost
		if(x == 0) segs[hidx[u]].update(vidx(u), INF);
		else segs[hidx[u]].update(vidx(u), vidx(u));
	}	
	int query(int v) { // root->v query
		return query_to(0, v);
	}	
	int query_to(int u, int v) { // return u->v path's query
		if(hidx[u] == hidx[v]) {
			int res = segs[hidx[u]].query(vidx(u), vidx(v)+1);
			if(res == INF) return INF;
			return hvy[hidx[u]][res];
		}
		int res = query_to(u, par[hvy[hidx[v]][0]][0]);
		if(res != INF) return res;
		res = segs[hidx[v]].query(0, vidx(v)+1);
		if(res == INF) return INF;
		return hvy[hidx[v]][res];
	}
};
\end{minted}
\subsection {HLD (Edge)}
\begin{minted}{cpp}
/*
	HLD with costed edge.
	Unlike the normal HLD, top edge of each chains are also belongs to chain.

	usually (dfs_init, lca, decomposite, eidx, query) don't need to be changed.
	just modify (segtree, init_segs), and if segtree function changed modify (update, query_to)
*/
const int LOGN = 18;
struct hld_edge {
	struct segtree { // just edit segtree ( currently half-open interval [s, e) )
		int n;
		vi t;
		void init(int _n) { n = _n; t.assign(2*n, 0); }
		void update(int u, int x) {
			for(t[u += n] = x; u > 1; u /= 2) t[u/2] = max(t[u], t[u^1]);
		}
		int query(int s, int e) {
			int ret = 0;
			for(s += n, e += n; s < e; s /= 2, e /= 2) {
				if(s&1) ret = max(ret, t[s++]);
				if(e&1) ret = max(ret, t[--e]);
			}
			return ret;
		}
	};
	int n, rt;
	vi ssz, dep, hidx;
	vvi g, par, hvy;
	vector<segtree> segs;
	hld_edge(vvi& g, int rt) : g(g), rt(rt), n(g.size()), ssz(n, 0), dep(n, 0), hidx(n, -1), par(n, vi(LOGN, 0)) { 
		par[rt][0] = rt;
		dfs_init(rt);
		decomposite(rt);
		init_segs();
	}
	void dfs_init(int u) { // initialize dfs info
		ssz[u] = 1;
		FOR(j, 1, LOGN) par[u][j] = par[par[u][j-1]][j-1];
		for(int v : g[u]) {
			if(par[u][0] == v) continue;
			par[v][0] = u;
			dep[v] = dep[u] + 1;
			dfs_init(v);
			ssz[u] += ssz[v];
		}
	}
	int lca(int u, int v) { // consider par[root] = root
		if(dep[u] < dep[v]) swap(u, v);
		int dif = dep[u] - dep[v];
		FOR(j, 0, LOGN) if(dif & (1<<j)) u = par[u][j];
		if(u != v) {
			RFOR(j, LOGN-1, 0) if(par[u][j] != par[v][j]) u = par[u][j], v = par[v][j];
			u = par[u][0];
		}
		return u;
	}
	void decomposite(int rt) { // decomposite tree
		hidx[rt] = -1;
		queue<int> q;
		q.push(rt);
		while(q.size()) {
			int u = q.front(); q.pop();
			for(int v : g[u]) if(par[v][0] == u) q.push(v);
			if(u != rt) {
				int p = par[u][0];
				if(p != rt && ssz[u]*2 >= ssz[p]) { // extend h-path (only if h-path)
					hidx[u] = hidx[p];
					hvy[hidx[u]].pb(u);
				} else { // create h-path (l-path or root-h-path)
					hidx[u] = hvy.size();
					hvy.pb(vi());
					hvy[hidx[u]].pb(p);
					hvy[hidx[u]].pb(u);
				}
			}
		}
	}
	void init_segs() { // initialize segtrees
		segs.assign(hvy.size(), segtree());
		FOR(i, 0, hvy.size()) segs[i].init(hvy[i].size()-1); // m vertices: m-1 edges
	}
	int eidx(int v) { // get u->v edge index in h-path
		return dep[par[v][0]] - dep[hvy[hidx[v]][0]];
	}
	void update(int u, int v, int x) { // u->v edge update
		if(par[u][0] == v) swap(u, v);
		assert(par[v][0] == u);
		segs[hidx[v]].update(eidx(v), x);
	}
	int query_to(int u, int v) { // return u->v path's query
		if(u == v) return 0;
		// modify range if segtree use closed interval [s, e]
		if(hidx[u] == hidx[v]) return segs[hidx[u]].query(eidx(u)+1, eidx(v)+1); // e(u)+1 because target is edge
		return max(query_to(u, hvy[hidx[v]][0]), segs[hidx[v]].query(0, eidx(v)+1)); // query tail path + recur
	}
	int query(int u, int v) {
		int t = lca(u, v);
		return max(query_to(t, u), query_to(t, v));
	}
};
\end{minted}
\section{Miscellaneous}
\subsection {Preset}
\begin{minted}{cpp}
#include <bits/stdc++.h>
#include <ext/pb_ds/assoc_container.hpp>
#include <ext/pb_ds/tree_policy.hpp>
using namespace std;
using namespace __gnu_pbds;
template<class key, class value, class cmp = less<key>>
using treemap = tree<key, value, less<int>, rb_tree_tag, tree_order_statistics_node_update>; // key, val, comp, implements, 노드 불변 규칙
template<class key, class cmp = less<key>>
using treeset = tree<key, null_type, cmp, rb_tree_tag, tree_order_statistics_node_update>;
#ifdef LOCAL_BOOKNU
#define debug(...) cerr << "[" << #__VA_ARGS__ << "]:", debug_out(__VA_ARGS__)
#else
#define debug(...) 42
#endif
#define FOR(i, f, n) for(int (i) = (f); (i) < (int)(n); ++(i))
#define RFOR(i, f, n) for(int (i) = (f); (i) >= (int)(n); --(i))
#define pb push_back
#define emb emplace_back
#define fi first
#define se second
#define ENDL '\n'
#define sz(A) (int)(A).size()
#define ALL(A) A.begin(), A.end()
#define UNIQUE(c) (c).resize(unique(ALL(c)) - (c).begin())
typedef pair<int, int> ii;
typedef pair<int, ii> iii;
typedef vector<int> vi;
typedef vector<vi> vvi;
typedef vector<ii> vii;
typedef vector<vii> vvii;
typedef long long i64;
typedef unsigned long long ui64;
inline int getidx(const vi& ar, int x) { return lower_bound(ALL(ar), x) - ar.begin(); }
inline i64 GCD(i64 a, i64 b) { i64 n; if(a < b) swap(a, b); while(b != 0) { n = a % b; a = b; b = n; } return a; }
inline i64 LCM(i64 a, i64 b) { if(a == 0 || b == 0) return GCD(a, b); return a / GCD(a, b) * b; }
inline i64 CEIL(i64 n, i64 d) { return n / d + (i64)(n % d != 0); } // for positive numbers
inline i64 ROUND(i64 n, i64 d) { return n / d + (i64)((n % d) * 2 >= d); }
const i64 MOD = 1e9+7;
inline i64 POW(i64 a, i64 n) {
	i64 ret;
	for(ret = 1; n; a = a*a%MOD, n /= 2) { if(n%2) ret = ret*a%MOD; }
	return ret;
}
template <class T> ostream& operator<<(ostream& os, vector<T> v) {
	os << "[";
	int cnt = 0;
	for(auto vv : v) { os << vv; if(++cnt < v.size()) os << ","; }
	return os << "]";
}
template <class T> ostream& operator<<(ostream& os, set<T> v) {
	os << "[";
	int cnt = 0;
	for(auto vv : v) { os << vv; if(++cnt < v.size()) os << ","; }
	return os << "]";
}
template <class L, class R> ostream& operator<<(ostream& os, pair<L, R> p) { return os << "(" << p.fi << "," << p.se << ")"; }
void debug_out() { cerr << endl; }
template <typename Head, typename... Tail> void debug_out(Head H, Tail... T) { cerr << " " << H, debug_out(T...); }
// ........................... MAIN ........................... //   
void input() { }
int solve() { return 0; }
void execute() { input(), solve(); }
int main(void) {
#ifdef LOCAL_BOOKNU
	freopen("__IO/input.txt", "r", stdin);
	// freopen("__IO/out.txt", "w", stdout);
#endif
	cin.tie(0), ios_base::sync_with_stdio(false);
	execute();
	return 0;
}
\end{minted}
\subsection {3D-Partial Sum}
\begin{minted}{cpp}
const int RANGE = 256;
int n, k, ps[RANGE][RANGE][RANGE], ar[RANGE][RANGE][RANGE];
int f(int x, int y, int z) {
	return (x < 0 || y < 0 || z < 0 || x >= RANGE || y >= RANGE || z >= RANGE ? 0 : ps[x][y][z]);
}
int sum(int x1, int y1, int z1, int l) {
	int x2 = min(RANGE - 1, x1 + l), y2 = min(RANGE - 1, y1 + l), z2 = min(RANGE - 1, z1 + l);
	--x1, --y1, --z1;
	return
		f(x2, y2, z2)
		- f(x1, y2, z2) - f(x2, y1, z2) - f(x2, y2, z1)
		+ f(x1, y1, z2) + f(x1, y2, z1) + f(x2, y1, z1)
		- f(x1, y1, z1);
}
void init() {
	FOR(x, 0, RANGE) {
		FOR(y, 0, RANGE) {
			FOR(z, 0, RANGE) {
				ps[x][y][z] +=
					f(x - 1, y, z) + f(x, y - 1, z) + f(x, y, z - 1)
					- f(x - 1, y - 1, z) - f(x - 1, y, z - 1) - f(x, y - 1, z - 1)
					+ f(x - 1, y - 1, z - 1);
			}
		}
	}
}
\end{minted}
\subsection {Knuth's Optimization}
If three conditions satisfies in DP equation, the time complexity can be reduced from $O(n^3)$ to $O(n^2)$
\begin{enumerate}
    \item DP Equation Form
        \begin{itemize}
            \item $D[i][j] = \min_{i < k < j}(D[i][k] + D[k][j]) + C[i][j]$
        \end{itemize}
    \item Quadrangle Inequality
        \begin{itemize}
            \item $C[a][c] + C[b][d] \leq C[a][d] + C[b][c], a \leq b \leq c \leq d$
        \end{itemize}
    \item Monotonicity
        \begin{itemize}
            \item $C[b][c] \leq C[a][d], a \leq b \leq c \leq d$
        \end{itemize}
\end{enumerate}
Define $A[i][j] = k \text{ for } D[i][j] \text{ becomes minimum}$ \\
If condition $2$, $3$ been satisfied, bellowing inequality holds. \\
$A[i][j-1] \leq A[i][j] \leq A[i+1][j]$
\subsection {Bit Tricks}
\begin{minted}{cpp}
__builtin_clz(int x); // count leading-zero
__builtin_ctz(int x); // count tailing-zero
__builtin_clzll(i64 x);
__builtin_ctzll(i64 x);
__builtin_popcount(int x); // number of 1-bits
__builtin__ffs(int x); // lsb index (1-based, x = 0 -> 0)

floor(log2(n)): 31 - __builtin_clz(n|1);
// 00111, 01011, 01101, 01110, 10011, 10101...
i64 next_perm(i64 x) {
    i64 t = x|(x-1);
    return (t + 1) | (((~t & -~t) - 1) >> (__builtin_ctz(x) + 1))
}
\end{minted}
\subsection {Fast IO}
\begin{minted}{cpp}
class FastIO {
	int fd, bufsz;
	char *buf, *cur, *end;
public:
	FastIO(int _fd = 0, int _bufsz = 1 << 20) : fd(_fd), bufsz(_bufsz) {
		buf = cur = end = new char[bufsz];
	}
	~FastIO() { delete[] buf; }
	bool readbuf() {
		cur = buf;
		end = buf + bufsz;
		while(true) {
			size_t res = fread(cur, sizeof(char), end - cur, stdin);
			if(res == 0) break;
			cur += res;
		}
		end = cur;
		cur = buf;
		return buf != end;
	}
	bool hasNext() {
		while(true) {
			if(cur == end && !readbuf()) return false;
			if(isdigit(*cur) || *cur == '-') break;
			++cur;
		}
		return true;
	}
	int r() {
		while(true) {
			if(cur == end) readbuf();
			if(isdigit(*cur) || *cur == '-') break;
			++cur;
		}
		bool sign = true;
		if(*cur == '-') {
			sign = false;
			++cur;
		}
		int ret = 0;
		while(true) {
			if(cur == end && !readbuf()) break;
			if(!isdigit(*cur)) break;
			ret = ret * 10 + (*cur - '0');
			++cur;
		}
		return sign ? ret : -ret;
	}
} sc;
\end{minted}
\subsection {Input Format}
\begin{minted}{cpp}
while(scanf("%d", &n) > 0) { // until input ends
    scanf("%d: (%d)", &x, &y); // formatted input
    for(int i = 0; i < x; ++i) scanf("%d", &z);
}
getline(cin, line);
\end{minted}
\subsection {String Parsing}
\begin{minted}{cpp}
vector<string> split(string& target, string regex) { // using regex
	vector<string> ret;
	std::regex rgx(regex);
	std::sregex_token_iterator iter(target.begin(),
		target.end(),
		rgx,
		-1);
	std::sregex_token_iterator end;
	for( ; iter != end; ++iter) ret.pb(*iter);
	return ret;
}
vector<string> space_split(string& s) { // split by whitespace 
	istringstream iss(s);
	vector<string> ret(istream_iterator<string>{iss}, istream_iterator<string>());
	return ret;
}
std::to_string(42);
std::atoi("42");
\end{minted}
\subsection {Ordered Statistics Tree in g++}
\begin{minted}{cpp}
s.find_by_order(x); // 0-indexed
s.order_of_key(x) // 0-indexed, find first element x <= ar[idx]
\end{minted}
\subsection {Prime Numbers}
\begin{minted}{cpp}
	< 10^k          number     divisors   2 3 5 71113171923293137
	-------------------------------------------------------------
	1                    6            4   1 1
	2                   60           12   2 1 1
	3                  840           32   3 1 1 1
	4                 7560           64   3 3 1 1
	5                83160          128   3 3 1 1 1
	6               720720          240   4 2 1 1 1 1
	7              8648640          448   6 3 1 1 1 1
	8             73513440          768   5 3 1 1 1 1 1
	9            735134400         1344   6 3 2 1 1 1 1
	10          6983776800         2304   5 3 2 1 1 1 1 1
	11         97772875200         4032   6 3 2 2 1 1 1 1
	12        963761198400         6720   6 4 2 1 1 1 1 1 1
	13       9316358251200        10752   6 3 2 1 1 1 1 1 1 1
	14      97821761637600        17280   5 4 2 2 1 1 1 1 1 1
	15     866421317361600        26880   6 4 2 1 1 1 1 1 1 1 1
	16    8086598962041600        41472   8 3 2 2 1 1 1 1 1 1 1
	17   74801040398884800        64512   6 3 2 2 1 1 1 1 1 1 1 1
	18  897612484786617600       103680   8 4 2 2 1 1 1 1 1 1 1 1

	< 10^k    prime   # of prime          < 10^k            prime
	-------------------------------------------------------------
	1             7            4          10           9999999967
	2            97           25          11          99999999977
	3           997          168          12         999999999989
	4          9973         1229          13        9999999999971
	5         99991         9592          14       99999999999973
	6        999983        78498          15      999999999999989
	7       9999991       664579          16     9999999999999937
	8      99999989      5761455          17    99999999999999997
	9     999999937     50847534          18   999999999999999989
\end{minted}
\subsection {Random}
\begin{minted}{cpp}
mt19937 rng(chrono::steady_clock::now().time_since_epoch().count());
mt19937 rng(0x14004); // 0x94949
int randint(int s, int e) { return uniform_int_distribution<int>(s, e)(rng); }
\end{minted}
\subsection {Hashing}
\begin{minted}{cpp}
struct chash {
	const int RANDOM = (long long)(make_unique<char>().get()) ^ chrono::high_resolution_clock::now().time_since_epoch().count();
	static unsigned long long hash_f(unsigned long long x) {
		x += 0x9e3779b97f4a7c15;
		x = (x ^ (x >> 30)) * 0xbf58476d1ce4e5b9;
		x = (x ^ (x >> 27)) * 0x94d049bb133111eb;
		return x ^ (x >> 31);
	}
	static unsigned hash_combine(unsigned a, unsigned b) { return a * 31 + b; }
	int operator()(int x) const { return hash_f(x)^RANDOM; }
};

gp_hash_table<key, int, chash> mp;
int main() {
	mp[1] = 1;
}
\end{minted}
\end{document}